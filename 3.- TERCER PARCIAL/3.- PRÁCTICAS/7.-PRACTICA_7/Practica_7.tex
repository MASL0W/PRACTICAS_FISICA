\documentclass[12pt,a4paper]{article}
\input{preamble.tex}
\begin{document}

\portada

\section{RESUMEN.} % (((
% )))

\section{INTRODUCCIÓN.} % (((
Si el ángulo es pequeño, entonces el \(OPQ\) es semejante a \(S_1S_1 \delta\).
\begin{figure}[hbtp!]
	\centering
	\includegraphics[width= 0.8 \linewidth]{1_INTRO/geometria.png}
	\caption{Experimento de la Doble Rendija de Young.}
	\label{fig:constructiva}
\end{figure}\\
La interferencia es constructiva siempre que \(r_1-r_2 = m \lambda\), para algún \(m \in \mathds{Z}\). Es decir
\begin{equation}
	d \sin \theta = m \lambda , \hspace{1cm} m \in \mathds{Z} .
	\label{eq:young}
\end{equation}
donde se alcanzará un máximo cuando \(m \in \mathds{Z}\). Y si \(m=k/2\), con \(k\) impar, entonces se alcanzará un mínimo. Además, el campo eléctrico \(E_i\) en el punto \(S_i\) está dado por
\begin{equation}
	\begin{array}{rcl}
		E_1 & = & E_0 \sin wt \\[2mm]
		E_2 & = & E_0 \sin (wt+ \phi)
	\end{array}
	\label{eq:campos_electricos}
\end{equation}
Y por el principio de superposición, tendremos que el campo eléctrico total es:
\begin{equation}
	E = E_1 + E_2 = (2E_0 \cos \beta) \sin (wt+ \beta).
	\label{eq:campo_total}
\end{equation}
donde \(\beta = \phi /2\). Recordamos que \(\beta = \dfrac{2 \pi \sin \theta}{\lambda}\), y la intensidad para el ángulo \(\theta\) está dada por
\begin{equation}
	I(\theta) = 4I_0 \cos ^2 \beta = 4I_0 \cos ^2 \bigg(\dfrac{\pi d \sin \theta}{\lambda}\bigg).
	\label{eq:intensidad}
\end{equation}
Los máximos son alcnazados cuando
\begin{equation}
	W \sin \theta = n \lambda , \hspace{1cm} n \in \mathds{Z}
	\label{eq:maximos}
\end{equation}
y análogamente al caso anterior, se tiene que
\begin{equation}
	\theta = \arctan (y/L).
	\label{eq:theta}
\end{equation}
% )))

\section{METODOLOGÍA.} % (((

\subsection{--- Difracción de una Rendija Simple y Doble Rendija ---} % (((
\label{sub:difraccion_simple}

\subsubsection{Rendija Simple.} % (((
\label{subs:rendija_simple}
% )))

\subsubsection{Doble Rendija.} % (((
\label{subs:doble_rendija}
% )))


% )))

\subsection{--- Rejilla de Difracción. ---} % (((
\label{sub:reijlla_de_difraccion}
% )))

% )))

\section{RESULTADOS.} % (((

	 La incertidumbre de la regla y la cinta métrica empleados era de $ \dfrac{0.1 cm}{2}=0.05 cm $
	 
	 Para la primera parte del experimento, donde se emplearon los filtros de colores, se obtuvieron las siguientes mediciones mostradas en la tabla \ref{tab:1}.
	 
	 \begin{table}[!htb]
	 	\centering
	 	\caption{Difracción en una rendija simple}
	 	\begin{tabular}{|c|c|c|c|c|}
	 		\hline
	 		Color & $ n $ & $ W $ (mm) & $ y\mbox{ }(\pm0.05\mbox{ cm})$  & L $(\pm0.05\mbox{ cm})$ \\
	 		\hline
	 		Rojo & 2 & 0.04 & 1.25 & 25 \\
	 		\hline
	 		Verde & 2 & 0.04 & 1 & 25 \\
	 		\hline
	 		Azul & 2 & 0.04 & 0.9 & 25 \\
	 		\hline
	 	\end{tabular}
	 	\label{tab:1}
	 \end{table}
	 
	 Entonces, en la tabla \ref{tab:2} se encuentran los resultados para la estimación de la longitud de onda de la luz empleada
	 
	 \begin{table}[!htb]
	 	\centering
	 	\caption{Estimación de $ \lambda $}
	 	\begin{tabular}{|c|c|c|}
	 		\hline
	 		Color & $ \theta $ (rad) & $ \lambda $ (nm)\\
	 		\hline
	 		Rojo & $ 0.0499\pm 0.0020 $ & $ 998.7523\pm 41.8429 $ \\
	 		\hline
	 		Verde & $ 0.0399\pm 0.0020 $ & $ 799.3608\pm 41.5003 $ \\
	 		\hline
	 		Azul & $ 0.0359\pm 0.0020 $ & $ 719.5339\pm 41.3595 $ \\
	 		\hline
	 	\end{tabular}
 		\label{tab:2}
	 \end{table}
	 
	 
	 Para el segundo experimento, se empleó la rejilla de difracción con 
	 $$ A=6000\dfrac{aberturas}{cm} $$ separada a una distancia $ L=(25\pm0.05)\mbox{ cm} $ tanto de la fuente como de la retina del ojo. Se obtuvieron entonces las siguientes mediciones, las cuales se muestran en la tabla \ref{tab:3}
	 
	 \begin{table}[!htb]
	 	\centering
	 	\caption{Datos para el cálculo del rango de $ \lambda $}
	 	\begin{tabular}{|c|c|c|}
	 		\hline
	 		Color & $ y_1\mbox{ }(\pm0.05\mbox{ cm})$ & $ y_2\mbox{ }(\pm0.05\mbox{ cm})$ \\ \hline
	 		Violeta    & 5.5 & 14   \\ \hline
	 		Azul       & 6.5 & 15.5 \\ \hline
	 		Verde      & 7   & 18   \\ \hline
	 		Amarillo   & 8   & 20   \\ \hline
	 		Anaranjado & 8.5 & 20.5 \\ \hline
	 		Rojo       & 10  & 23   \\ \hline
	 	\end{tabular}
	 	\label{tab:3}
	 \end{table}
	 
	 Empleando la fórmula \ref{}, se obtuvieron los siguientes resultados (tabla \ref{tab:4} \ref{tab:4.1})
	 
	 \begin{table}[!htb]
	 	\centering
	 	\caption{Resultados de la estimación de $ \lambda $ para cada color}
	 	\begin{tabular}{|l|c|c|c|}
	 		\hline
	 		Color & $ \lambda_1(\mbox{nm})$ & $ \lambda_2(\mbox{nm})$ & Promedio $ \lambda (\mbox{nm}) $ \\
	 		\hline
	 		Violeta    & $ 346.4805\pm3.7237 $ & $ 408.3907\pm1.9001 $ & $ 377.4356\pm2.8119 $ \\ \hline
	 		Azul       & $ 406.9889\pm3.7766 $ & $ 443.9966\pm1.8722 $ & $ 425.4927\pm2.8244 $ \\ \hline
	 		Verde      & $ 436.8139\pm3.7982 $ & $ 499.2184\pm1.8124 $ & $ 468.0162\pm2.8053 $ \\ \hline
	 		Amarillo   & $ 495.5247\pm3.8316 $ & $ 539.7928\pm1.7560 $ & $ 517.6587\pm2.7938 $ \\ \hline
	 		Anaranjado & $ 524.3816\pm3.8436 $ & $ 549.4541\pm1.7412 $ & $ 536.9179\pm2.7924 $ \\ \hline
	 		Rojo       & $ 608.8102\pm3.8620 $ & $ 595.0045\pm1.6637 $ & $ 601.9073\pm2.7629 $ \\ \hline
	 	\end{tabular}
	 	\label{tab:4}
	 \end{table}
	 
	 
	 
	 \begin{table}[!htb]
	 	\centering
	 	\caption{Resultados de la estimación de $ \lambda $ para cada color}
	 	\begin{tabular}{|l|c|c|c|}
	 		\hline
	 		Color & $ \lambda_1(\mbox{nm})$ & $ \lambda_2(\mbox{nm})$ & Promedio $ \lambda (\mbox{nm}) $ \\
	 		\hline
	 		\cellcolor{violet!40     }{Violeta   } & \cellcolor{violet!25     }{$ 346.4805\pm3.7237 $} & \cellcolor{violet!25     }{$ {408.3907\pm1.9001} $} & \cellcolor{violet!25     }{$ 377.4356\pm2.8119 $} \\ \hline
	 		\cellcolor{blue!40       }{Azul      } & \cellcolor{blue!25       }{$ 406.9889\pm3.7766 $} & \cellcolor{blue!25       }{$ {443.9966\pm1.8722} $} & \cellcolor{blue!25       }{$ 425.4927\pm2.8244 $} \\ \hline
	 		\cellcolor{ForestGreen!40}{Verde     } & \cellcolor{ForestGreen!25}{$ 436.8139\pm3.7982 $} & \cellcolor{ForestGreen!25}{$ {499.2184\pm1.8124} $} & \cellcolor{ForestGreen!25}{$ 468.0162\pm2.8053 $} \\ \hline
	 		\cellcolor{yellow!40     }{Amarillo  } & \cellcolor{yellow!25     }{$ 495.5247\pm3.8316 $} & \cellcolor{yellow!25     }{$ {539.7928\pm1.7560} $} & \cellcolor{yellow!25     }{$ 517.6587\pm2.7938 $} \\ \hline
	 		\cellcolor{orange!40     }{Anaranjado} & \cellcolor{orange!25     }{$ 524.3816\pm3.8436 $} & \cellcolor{orange!25     }{$ {549.4541\pm1.7412} $} & \cellcolor{orange!25     }{$ 536.9179\pm2.7924 $} \\ \hline
	 		\cellcolor{red!40        }{Rojo      } & \cellcolor{red!25        }{$ 608.8102\pm3.8620 $} & \cellcolor{red!25        }{$ {595.0045\pm1.6637} $} & \cellcolor{red!25        }{$ 601.9073\pm2.7629 $} \\ \hline
	 	\end{tabular}
	 	\label{tab:4.1}
	 \end{table}
	 
	 \newpage
	 Los rangos para la longitud de onda de los colores se muestran en la tabla \ref{tab:5} \ref{tab:5.1}
	 
	 \begin{table}[!htb]
	 	\centering
	 	\caption{Rangos para $ \lambda $ de cada color}
	 	\begin{tabular}{|l|c|}
	 		\hline
	 		Color & $ \lambda(\mbox{nm})$ \\
	 		\hline
	 		Violeta    & $ 400 - 455 $  \\ \hline
	 		Azul       & $ 455 - 490 $  \\ \hline
	 		Verde      & $ 490 - 570 $  \\ \hline
	 		Amarillo   & $ 570 - 590 $  \\ \hline
	 		Anaranjado & $ 590 - 620 $  \\ \hline
	 		Rojo       & $ 620 - 780 $  \\ \hline
	 	\end{tabular}
	 	\label{tab:5}
	 \end{table}
	 
	 \begin{table}[!htb]
	 	\centering
	 	\caption{Rangos para $ \lambda $ de cada color}
	 	\begin{tabular}{|l|c|}
	 		\hline
	 		Color & $ \lambda(\mbox{nm})$ \\
	 		\hline
	 		\cellcolor{violet!40     }{Violeta   } & \cellcolor{violet!25     }{$ 400 - 455 $}  \\ \hline
	 		\cellcolor{blue!40       }{Azul      } & \cellcolor{blue!25       }{$ 455 - 490 $}  \\ \hline
	 		\cellcolor{ForestGreen!40}{Verde     } & \cellcolor{ForestGreen!25}{$ 490 - 570 $}  \\ \hline
	 		\cellcolor{yellow!40     }{Amarillo  } & \cellcolor{yellow!25     }{$ 570 - 590 $}  \\ \hline
	 		\cellcolor{orange!40     }{Anaranjado} & \cellcolor{orange!25     }{$ 590 - 620 $}  \\ \hline
	 		\cellcolor{red!40        }{Rojo      } & \cellcolor{red!25        }{$ 620 - 780 $}  \\ \hline
	 	\end{tabular}
	 	\label{tab:5.1}
	 \end{table}
	 
	 Y a partir de estos valores, se obtuvieron los siguientes errores absolutos (calculados entre el valor promedio de $ \lambda $ y el mínimo valor de cada rango) mostrados en la tabla \ref{tab:6} \ref{tab:6.1}
	 
	 \begin{table}[!htb]
		 \centering
		 \caption{Cálculo del error absoluto para $ \lambda $}
		 \begin{tabular}{|l|c|}
		 	\hline
		 	Color & $ |\lambda_{\tiny\mbox{Estimado}}-\lambda_{\tiny\mbox{Tabla \ref{}}}| (\mbox{nm})$ \\
		 	\hline
		 	Violeta    & $ 22.5644 $  \\ \hline
		 	Azul       & $ 29.5073 $  \\ \hline
		 	Verde      & $ 21.9838 $  \\ \hline
		 	Amarillo   & $ 52.3416 $  \\ \hline
		 	Anaranjado & $ 53.0821 $  \\ \hline
		 	Rojo       & $ 18.0927 $  \\ \hline
		 \end{tabular}
		 \label{tab:6}
 	\end{table}
 
 
 	\begin{table}[!htb]
 		\centering
 		\caption{Cálculo del error absoluto para $ \lambda $}
 		\begin{tabular}{|l|c|}
 			\hline
 			Color & $ |\lambda_{\tiny\mbox{Estimado}}-\lambda_{\tiny\mbox{Tabla \ref{}}}| (\mbox{nm})$ \\
 			\hline
 			\cellcolor{violet!40     }{Violeta   } & \cellcolor{violet!25     }{$ 22.5644 $}  \\ \hline
 			\cellcolor{blue!40       }{Azul      } & \cellcolor{blue!25       }{$ 29.5073 $}  \\ \hline
 			\cellcolor{ForestGreen!40}{Verde     } & \cellcolor{ForestGreen!25}{$ 21.9838 $}  \\ \hline
 			\cellcolor{yellow!40     }{Amarillo  } & \cellcolor{yellow!25     }{$ 52.3416 $}  \\ \hline
 			\cellcolor{orange!40     }{Anaranjado} & \cellcolor{orange!25     }{$ 53.0821 $}  \\ \hline
 			\cellcolor{red!40        }{Rojo      } & \cellcolor{red!25        }{$ 18.0927 $}  \\ \hline
 		\end{tabular}
 		\label{tab:6.1}
 	\end{table}
	  
	 \newpage 





\section{DISCUSIÓN DE RESULTADOS Y CONCLUSIONES.} % (((
% )))

% === REFERENCIAS === (((
% \bibliography{Referencias}
% \bibliographystyle{unsrt}
% )))

\section{APÉNDICE.} % (((
% )))

	\subsection{--- Propagación de la Incertidumbre ---}
	 
	 La propagación de la incertidumbre para un producto y un cociente están dadas, respectivamente por
	 \begin{align*}
	 	(x\pm\delta x)(y\pm\delta y)&=x\cdot y\pm\left(|y|\delta x+|x|\delta y \right)\\\\
	 	\dfrac{x\pm\delta x}{y\pm\delta y}&=\dfrac{x}{y}\pm\left(\dfrac{\delta x}{|y|}+|x|\dfrac{\delta y}{|y|^2}\right)
	 \end{align*}
 
 	
 	Ahora, recordar que si $f$ es una función de $\mathbb{R}^n$ a $\mathbb{R}$ en la cual se le pueden medir sus variables (cada una asociada con su respectiva incertidumbre)
 	$$f(x_1\pm\Delta x_1,x_2\pm\Delta x_2,...,x_n\pm\Delta x_n)$$
 	entonces la propagación de la incertidumbre está dada por 
 	$$\Delta f=\pm\left(\displaystyle\sum_{i=1}^k\Delta x_i\cdot\left|\dfrac{\partial f(x_i)}{\partial x_i}\right| \right)$$
 	
 	En esta práctica se emplearán las siguientes
 	
 	\begin{itemize}
 		\item [$\cdot$] Función $ sen(\theta) $
 		
 		En este caso
 		
 		$$f(x\pm\Delta x)=sen(\theta\pm\Delta\theta)\Longrightarrow \Delta f=\Delta\theta\cdot\left|\dfrac{d sen(\theta)}{d\theta}\right|=\Delta\theta\cdot|cos(\theta)|$$
 		
 		Por lo que 
 		$$sen(\theta\pm\Delta\theta)=sen(\theta)\pm\Delta\theta\cdot|cos(\theta)|$$
 		
 		\item [$\cdot$] Función $ arctan(\theta) $
 		
 		Se tiene que 
 		
 		$$f(x\pm\Delta x)=arctan(\theta\pm\Delta\theta)\Longrightarrow \Delta f=\Delta\theta\cdot\left|\dfrac{d \mbox{ }arctan(\theta)}{d\theta}\right|=\Delta\theta\cdot\left|\dfrac{1}{\theta^2 + 1 }\right|$$
 		
 		Y entonces 
 		$$arctan(\theta\pm\Delta\theta)=arctan(\theta)\pm \dfrac{\Delta\theta}{\theta^2 + 1 }$$
 		
 	\end{itemize} 
 	
 	\newpage
 	\subsection{--- Estimación de la longitud de onda ---}
 	
 	A partir de los datos de la tabla \ref{tab:1} y empleando lo descrito al inicio del apéndice junto con las fórmulas \ref{} y \ref{}, se tiene lo siguiente.
 	
 	\begin{itemize}
 		\item [$\cdot$] Filtro Rojo
 		
 		\begin{align*}
 			\dfrac{y}{L}=\dfrac{(1.25\pm 0.05)\mbox{ cm}}{(25\pm 0.05)\mbox{ cm}}&=
 			\dfrac{1.25}{25}\pm\left(\dfrac{0.05}{25}+1.25\cdot\dfrac{0.05}{25^2}\right)=
 			0.05 \pm 0.0021\\\\
 			\Longrightarrow arctan(	0.05 \pm 0.0021)&=
 			arctan(0.05)\pm\dfrac{0.0021}{(0.05)^2+1}=0.0499\pm 0.0020\\\\
 			\Longrightarrow sen(0.0499\pm 0.0020)&=sen(0.0499)\pm 0.0020\cdot cos(0.0499)=0.0499\pm0.0020
 		\end{align*}
 		$$\Longrightarrow \lambda=\dfrac{(0.04\mbox{ mm})\cdot(0.0499\pm0.0020)}{2}=
 		(998.7523\pm 41.8429)\mbox{ nm}$$
 		
 		

 		\item [$\cdot$] Filtro Verde

		\begin{align*}
			\dfrac{y}{L}=\dfrac{(1\pm 0.05)\mbox{ cm}}{(25\pm 0.05)\mbox{ cm}}&=
			\dfrac{1}{25}\pm\left(\dfrac{0.05}{25}+1\cdot\dfrac{0.05}{25^2}\right)=
			0.04 \pm 0.0020\\\\
			\Longrightarrow arctan(	0.04 \pm 0.0020)&=
			arctan(0.04)\pm\dfrac{0.0020}{(0.04)^2+1}=
			0.0399\pm 0.0020\\\\
			\Longrightarrow sen(0.0399\pm 0.0020)&=sen(0.0399)\pm 0.0020\cdot cos(0.0399)=0.0399\pm 0.0020
		\end{align*}
		$$\Longrightarrow \lambda=\dfrac{(0.04\mbox{ mm})\cdot(0.0399\pm 0.0020)}{2}=
		(799.3608\pm 41.5003)\mbox{ nm}$$
		
 		\item [$\cdot$] Filtro Azul
 		
 		\begin{align*}
 			\dfrac{y}{L}=\dfrac{(0.9\pm 0.05)\mbox{ cm}}{(25\pm 0.05)\mbox{ cm}}&=
 			\dfrac{0.9}{25}\pm\left(\dfrac{0.05}{25}+0.9\cdot\dfrac{0.05}{25^2}\right)=
 			0.036 \pm 0.0020\\\\
 			\Longrightarrow arctan(0.036 \pm 0.0020)&=
 			arctan(0.036)\pm\dfrac{0.0020}{(0.036)^2+1}=
 			0.0359\pm 0.0020\\\\
 			\Longrightarrow sen(0.0359\pm 0.0020)&=sen(0.0359)\pm 0.0020\cdot cos(0.0359)=0.0359\pm 0.0020
 		\end{align*}
 		$$\Longrightarrow \lambda=\dfrac{(0.04\mbox{ mm})\cdot(0.0359\pm 0.0020)}{2}=
 		(719.5339\pm 41.3595)\mbox{ nm}$$

 	\end{itemize}
 	
 	\newpage
 	\subsection{--- Estimación del rango de la longitud de onda ---}
 	
 	\begin{itemize}
 		\item [$\cdot$] Color Violeta
 		
 		\begin{align*}
 			\dfrac{y_1}{L}=\dfrac{(5.5\pm 0.05)\mbox{ cm}}{(25\pm 0.05)\mbox{ cm}}&=
 			\dfrac{5.5}{25}\pm\left(\dfrac{0.05}{25}+5.5\cdot\dfrac{0.05}{25^2}\right)=
 			0.2200\pm0.0024 \\\\
 			\Longrightarrow arctan(0.22\pm0.0024)&=
 			arctan(0.22)\pm\dfrac{0.0024}{(0.22)^2+1}=
 			0.2165\pm 0.0023
 		\end{align*}
 		$$\Longrightarrow \lambda_1=
 		(1.6\times 10^{-4}\mbox{ cm})\cdot
 		(0.2165\pm 0.0023)=(346.4805\pm 3.7237)\mbox{ nm}$$
 		
 		\begin{align*}
 			\dfrac{y_2}{L}=\dfrac{(14\pm 0.05)\mbox{ cm}}{(25\pm 0.05)\mbox{ cm}}&=
 			\dfrac{14}{25}\pm\left(\dfrac{0.05}{25}+14\cdot\dfrac{0.05}{25^2}\right)=
 			0.5600\pm0.0031 \\\\
 			\Longrightarrow arctan(0.5600\pm0.0031)&=
 			arctan(0.5600)\pm\dfrac{0.0031}{(0.5600)^2+1}=
 			0.5104\pm 0.0023
 		\end{align*}
 		$$\Longrightarrow \lambda_2=
 		\left(\dfrac{1.6\times 10^{-4}\mbox{ cm}}{2}\right)\cdot
 		(0.5104\pm 0.0023 )=(408.3907\pm1.9001 )\mbox{ nm}$$
 		
 		
 		\item [$\cdot$] Color Azul
 		
 		\begin{align*}
 			\dfrac{y_1}{L}=\dfrac{(6.5\pm 0.05)\mbox{ cm}}{(25\pm 0.05)\mbox{ cm}}&=
 			\dfrac{6.5}{25}\pm\left(\dfrac{0.05}{25}+6.5\cdot\dfrac{0.05}{25^2}\right)=
 			0.2600\pm0.0025 \\\\
 			\Longrightarrow arctan(0.2600\pm0.0025)&=
 			arctan(0.2600)\pm\dfrac{0.0025}{(0.2600)^2+1}=
 			0.2543\pm0.0023 
 		\end{align*}
 		$$\Longrightarrow \lambda_1=
 		(1.6\times 10^{-4}\mbox{ cm})\cdot
 		(0.2543\pm0.0023 )=(406.9889\pm3.7766)\mbox{ nm}$$
 		
 		\begin{align*}
 			\dfrac{y_2}{L}=\dfrac{(15.5\pm 0.05)\mbox{ cm}}{(25\pm 0.05)\mbox{ cm}}&=
 			\dfrac{15.5}{25}\pm\left(\dfrac{0.05}{25}+15.5\cdot\dfrac{0.05}{25^2}\right)=
 			0.6200\pm0.0032 \\\\
 			\Longrightarrow arctan(0.6200\pm0.0032)&=
 			arctan(0.6200)\pm\dfrac{0.0032}{(0.6200)^2+1}=
 			0.5549\pm 0.0023
 		\end{align*}
 		$$\Longrightarrow \lambda_2=
 		\left(\dfrac{1.6\times 10^{-4}\mbox{ cm}}{2}\right)\cdot
 		(0.5549\pm 0.0023 )=(443.9966\pm 1.8722)\mbox{ nm}$$
 		
 		\item [$\cdot$] Color Verde
 		
 		\begin{align*}
 			\dfrac{y_1}{L}=\dfrac{(7\pm 0.05)\mbox{ cm}}{(25\pm 0.05)\mbox{ cm}}&=
 			\dfrac{7}{25}\pm\left(\dfrac{0.05}{25}+7\cdot\dfrac{0.05}{25^2}\right)=
 			0.2800\pm0.0025 \\\\
 			\Longrightarrow arctan(0.2800\pm0.0025)&=
 			arctan(0.2800)\pm\dfrac{0.0025}{(0.280)^2+1}=
 			0.2730\pm0.0023 
 		\end{align*}
 		$$\Longrightarrow \lambda_1=
 		(1.6\times 10^{-4}\mbox{ cm})\cdot
 		(0.2730\pm0.0023)=(436.8139\pm3.7982 )\mbox{ nm}$$
 		
 		\begin{align*}
 			\dfrac{y_2}{L}=\dfrac{(18\pm 0.05)\mbox{ cm}}{(25\pm 0.05)\mbox{ cm}}&=
 			\dfrac{18}{25}\pm\left(\dfrac{0.05}{25}+18\cdot\dfrac{0.05}{25^2}\right)=
 			0.7200\pm0.0034 \\\\
 			\Longrightarrow arctan(0.7200\pm0.0034)&=
 			arctan(0.7200)\pm\dfrac{0.0034}{(0.7200)^2+1}=
 			0.6240\pm0.0022
 		\end{align*}
 		$$\Longrightarrow \lambda_2=
 		\left(\dfrac{1.6\times 10^{-4}\mbox{ cm}}{2}\right)\cdot
 		(0.6240\pm0.0022 )=(499.2184\pm 1.8124)\mbox{ nm}$$
 		
 		\item [$\cdot$] Color Amarillo
 		
 		\begin{align*}
 			\dfrac{y_1}{L}=\dfrac{(8\pm 0.05)\mbox{ cm}}{(25\pm 0.05)\mbox{ cm}}&=
 			\dfrac{8}{25}\pm\left(\dfrac{0.05}{25}+8\cdot\dfrac{0.05}{25^2}\right)=
 			0.3200\pm0.0026 \\\\
 			\Longrightarrow arctan(0.3200\pm0.0026)&=
 			arctan(0.3200)\pm\dfrac{0.0026}{(0.3200)^2+1}=
 			0.3097\pm 0.0023
 		\end{align*}
 		$$\Longrightarrow \lambda_1=
 		(1.6\times 10^{-4}\mbox{ cm})\cdot
 		(0.3097\pm 0.0023 )=(495.5247\pm3.8316 )\mbox{ nm}$$
 		
 		\begin{align*}
 			\dfrac{y_2}{L}=\dfrac{(20\pm 0.05)\mbox{ cm}}{(25\pm 0.05)\mbox{ cm}}&=
 			\dfrac{20}{25}\pm\left(\dfrac{0.05}{25}+20\cdot\dfrac{0.05}{25^2}\right)=
 			0.8000\pm0.0036 \\\\
 			\Longrightarrow arctan(0.8000\pm0.0036)&=
 			arctan(0.8000)\pm\dfrac{0.0036}{(0.8000)^2+1}=
 			0.6747\pm0.0021 
 		\end{align*}
 		$$\Longrightarrow \lambda_2=
 		\left(\dfrac{1.6\times 10^{-4}\mbox{ cm}}{2}\right)\cdot
 		(0.6747\pm0.0021)=(539.7928\pm1.7560 )\mbox{ nm}$$
 		
 		
 		\item [$\cdot$] Color Anaranjado
 		
 		\begin{align*}
 			\dfrac{y_1}{L}=\dfrac{(8.5\pm 0.05)\mbox{ cm}}{(25\pm 0.05)\mbox{ cm}}&=
 			\dfrac{8.5}{25}\pm\left(\dfrac{0.05}{25}+8.5\cdot\dfrac{0.05}{25^2}\right)=
 			0.0500\pm0.0026 \\\\
 			\Longrightarrow arctan(0.0500\pm0.0026 )&=
 			arctan(0.0500)\pm\dfrac{0.0026}{(0.0500)^2+1}=
 			0.3277\pm 0.0024
 		\end{align*}
 		$$\Longrightarrow \lambda_1=
 		(1.6\times 10^{-4}\mbox{ cm})\cdot
 		(0.3277\pm 0.0024 )=(524.3816\pm3.8436 )\mbox{ nm}$$
 		
 		\begin{align*}
 			\dfrac{y_2}{L}=\dfrac{(20.5\pm 0.05)\mbox{ cm}}{(25\pm 0.05)\mbox{ cm}}&=
 			\dfrac{20.5}{25}\pm\left(\dfrac{0.05}{25}+20.5\cdot\dfrac{0.05}{25^2}\right)=
 			0.8200\pm0.0036 \\\\
 			\Longrightarrow arctan(0.8200\pm0.0036)&=
 			arctan(0.8200)\pm\dfrac{0.0036}{(0.8200)^2+1}=
 			0.6868\pm 0.0021
 		\end{align*}
 		$$\Longrightarrow \lambda_2=
 		\left(\dfrac{1.6\times 10^{-4}\mbox{ cm}}{2}\right)\cdot
 		(0.6868\pm 0.0021 )=(549.4541\pm 1.7412 )\mbox{ nm}$$
 		
 		\item [$\cdot$] Color Rojo
 		
 		\begin{align*}
 			\dfrac{y_1}{L}=\dfrac{(10\pm 0.05)\mbox{ cm}}{(25\pm 0.05)\mbox{ cm}}&=
 			\dfrac{10}{25}\pm\left(\dfrac{0.05}{25}+10\cdot\dfrac{0.05}{25^2}\right)=
 			0.4000\pm0.0028 \\\\
 			\Longrightarrow arctan(0.4000\pm0.0028)&=
 			arctan(0.4000)\pm\dfrac{0.0028}{(0.4000)^2+1}=
 			0.3805\pm0.0024 
 		\end{align*}
 		$$\Longrightarrow \lambda_1=
 		(1.6\times 10^{-4}\mbox{ cm})\cdot
 		(0.3805\pm0.0024 )=(608.8102\pm3.8620 )\mbox{ nm}$$
 		
 		\begin{align*}
 			\dfrac{y_2}{L}=\dfrac{(23\pm 0.05)\mbox{ cm}}{(25\pm 0.05)\mbox{ cm}}&=
 			\dfrac{23}{25}\pm\left(\dfrac{0.05}{25}+23\cdot\dfrac{0.05}{25^2}\right)=
 			0.9200\pm0.0038 \\\\
 			\Longrightarrow arctan(0.9200\pm0.0038)&=
 			arctan(0.9200)\pm\dfrac{0.0038}{(0.9200)^2+1}=
 			0.7437\pm 0.0020
 		\end{align*}
 		$$\Longrightarrow \lambda_2=
 		\left(\dfrac{1.6\times 10^{-4}\mbox{ cm}}{2}\right)\cdot
 		(0.7437\pm 0.0020 )=(595.0045\pm1.6637 )\mbox{ nm}$$
 		
 		
 	\end{itemize}
 	
 	


\end{document}
